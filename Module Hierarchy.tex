\newcounter{mnum}
\newcommand{\mthemnum}{M\themnum}
\newcommand{\mref}[1]{M\ref{#1}}

\section{Module Hierarchy} \label{SecMH}

Module Hierarchy provides the guidelines of the module design. Modules are sorted
in a hierarchy decomposed by secrets in Table \ref{TblMH}. The modules referred below are the modules that are being implemented and serve as the 'Leaf-nodes' in the hierarchy tree.

\begin{description}
\item [\refstepcounter{mnum} \mthemnum \label{mHH}:] Hardware-Hiding Module
\item [\refstepcounter{mnum} \mthemnum \label{mMain}:] Main Module
\item [\refstepcounter{mnum} \mthemnum \label{mUI}:] userInput Module
\item [\refstepcounter{mnum} \mthemnum \label{mPrintB}:] printBoard Module
\item [\refstepcounter{mnum} \mthemnum \label{mColor}:] getColor Module
\item [\refstepcounter{mnum} \mthemnum \label{mGetB}:] getBoard Module
\item [\refstepcounter{mnum} \mthemnum \label{mRemove}:] removeTile Module
\item [\refstepcounter{mnum} \mthemnum \label{mAdd}:] addTile Module
\item [\refstepcounter{mnum} \mthemnum \label{mMatch}:] colorMatch Module
\item [\refstepcounter{mnum} \mthemnum \label{mAdj}:] adjacent Module
\item [\refstepcounter{mnum} \mthemnum \label{mCol}:] checkColumn Module
\item [\refstepcounter{mnum} \mthemnum \label{mDown}:] moveDown Module

\end{description}

The Operating System (OS) must implement some modules in order to perform other functions. \mref{mHH} is a commonly used to refer to modules that have been implemented by the OS.  Therefore there is no requirement for them to be reimplemented. All the other modules listen below serve to provide an input/output or make logical decisions to implement the rules of Zop.
