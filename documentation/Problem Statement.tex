\documentclass[12pt]{article}
\begin{document}
\title{Problem Statement}
\maketitle

When playing games, the gamer can often come across various glitches, i.e. a lag in the game, that hamper the gaming experience. The project we have chosen is a game called Zop. Zop is along the lines of Candy Crush Saga and Bejeweled, wherein the player has to connect squares of like colour either vertically and/or horizontally, but not diagonally. Zop, like so many other games, has a few glitches. The primary glitch is that the squares, at times, do not stack up properly. Another glitch is when the game window is resized, the game becomes unresponsive. 

Puzzle games are liked by children and adults equally. Children at young ages use puzzles to develop their cognitive and emotional skills. Puzzles make players think and develop patience. Many adults who have a busy daily routine, look for stress relievers such as puzzle games. As developers, it is imperative for us to make sure the program is bug-free when complied. However, it is also our responsibility to improve the modularity of the program. By using modularization, the application developers can increase maintainability and reusability of the program as a whole.

The scope of the problem is directed toward anyone who is looking to pick up a quick, easy-to-understand game in their spare time. The final product will be very similar to the original product, however, the final product will have some major issues addressed and an improved user experience. 
The stakeholders of the project mainly include the group members, group supervisors, and the original creator of Zop.
As this project is treated as an academic exercise, we hope to get an invaluable experience in managing a small project as well as improving our knowledge of Git and LaTeX.




\end{document}
